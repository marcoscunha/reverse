\chapter {CPU Abstraction Layer}
This API is composed of 6 parts: \textit {endianness}, context management,
I/O management, Traps management, multiprocessor management and software
spinlocks management. Its implementations are located in the \texttt {cal} directory
of \textit {SSP}'s root directory.
\skipon

\skipoff
\section {\textit {Endianness} management}
This section presents the elements that allow to deal with \textit {endianness}
at a processor level.
\skipon

\begin{procedure}{CPU\_ENDIAN\_IS\_BIG\_\{16,32,64\}}{1}
	\pinout{value}{Integer\{16,32,64\}}{}
\end{procedure}

Depending on the processor's \textit{endianness}, this procedure
reorders the bytes of the \{16,32,64\} bits word \emph{value}
in a \textit{big-endian} fashion.

\begin{procedure}{CPU\_ENDIAN\_IS\_LITTLE\_\{16,32,64\}}{1}
	\pinout{value}{Integer\{16,32,64\}}{}
\end{procedure}

Depending on the processor's \textit{endianness}, this procedure
reorders the bytes of the \{16,32,64\} bits word \emph{value}
in a \textit{little-endian} fashion.

\begin{procedure}{CPU\_ENDIAN\_CONCAT}{4}
	\pin{size}{Integer32}{}\\
	\pout{result}{Integer\{16,32,64\}}{}\\
	\pin{low}{Integer\{8,16,32\}}{}\\
	\pin{high}{Integer\{8,16,32\}}{}
\end{procedure}

This procedure concatenates two \{8,16,32\} bits elements into one
\{16,32,64\} bits elements that matches the processor's endianness.

\begin{procedure}{CPU\_ENDIAN\_SPLIT}{4}
	\pin{size}{Integer32}{}\\
	\pin{value}{Integer\{16,32,64\}}{}\\
	\pout{low}{Integer\{8,16,32\}}{}\\
	\pout{high}{Integer\{8,16,32\}}{}
\end{procedure}

This procedure splits one \{16,32,64\} bits element matching the processor's
endianness into two \{8,16,32\} bits elements.

\skipoff
\section {Execution context management}
This section presents the elements that allow to deal with execution
contexts.
\skipon

\type{CPU\_CONTEXT\_T}

This type represents the execution context of the processor.

\definition{CPU\_CONTEXT\_SIZE}

This definition defines the context's size of the processor.

\begin{procedure}{CPU\_CONTEXT\_INIT}{5}
	\pout{context}{access CPU\_CONTEXT\_T}{}\\
	\pin{stack}{access array of Integer8}{}\\
	\pin{size}{Integer32}{The stack's size}\\
	\pin{entry}{access procedure}{}\\
	\pin{arguments}{access record}{}
\end{procedure}

This procedure initializes an execution context with a \emph{stack}
of a specific \emph{size}, a \emph{entry} point and some
\emph{arguments} that will be passed to the entry point. Note that
the stack must have been allocated.

\begin{procedure}{CPU\_CONTEXT\_LOAD}{1}
	\pin{context}{access CPU\_CONTEXT\_T}{}
\end {procedure}

This procedure loads a specific \emph{context}.

\begin{procedure}{CPU\_CONTEXT\_SAVE}{1}
	\pout{context}{access CPU\_CONTEXT\_T}{}
\end {procedure}

This procedure saves the current execution context into \emph{context}.

\begin{procedure}{CPU\_CONTEXT\_SWITCH}{2}
	\pout{from}{access CPU\_CONTEXT\_T}{}\\
	\pin{to}{access CPU\_CONTEXT\_T}{}
\end {procedure}

This procedure saves the current execution context into \emph{from}
and loads the context \emph{to}.

\skipoff
\section {I/O management}
This section presents the elements that allow to deal with I/Os.
\skipon

\begin{procedure}{CPU\_READ}{3}
	\pin{size}{Integer32}{}\\
	\pin{address}{access Integer\{8,16,32,64\}}{}\\
	\pout{result}{Integer\{8,16,32,64\}}{}
\end{procedure}

This procedure reads a \{8,16,32,64\} bits integer from \emph {address}
to \emph {result}.

\begin{procedure}{CPU\_READ\_UNCACHED}{3}
	\pin{size}{Integer32}{}\\
	\pin{address}{access Integer\{8,16,32,64\}}{}\\
	\pout{result}{Integer\{8,16,32,64\}}{}
\end{procedure}

This procedure reads a \{8,16,32,64\} bits integer from \emph {address}
to \emph {result} in a non-cached fashion.

\begin{procedure}{CPU\_READ\_VECTOR}{3}
	\pin{mode}{VectorMode}{Integer 8, 16, 32, 64; Float; Double}\\
	\pin{from}{access <VectorMode>}{}\\
	\pout{to}{access <VectorMode>}{}\\
	\pin{size}{Integer32}{}
\end{procedure}

This procedure reads a vector of \emph{size}, \emph{from} and writes
it \emph{to}. This operation is executed in a specific \emph{mode}.

\begin{procedure}{CPU\_WRITE}{3}
	\pin{size}{Integer32}{}\\
	\pin{address}{access Integer\{8, 16,32,64\}}{}\\
	\pout{value}{Integer\{8, 16,32,64\}}{}
\end{procedure}

This procedure writes a \{8, 16,32,64\} bits integer from \emph {value}
to \emph {address}.

\begin{procedure}{CPU\_WRITE\_UNCACHED}{3}
	\pin{size}{Integer32}{}\\
	\pin{address}{access Integer\{8, 16,32,64\}}{}\\
	\pout{value}{Integer\{8, 16,32,64\}}{}
\end{procedure}

This procedure reads a \{8, 16,32,64\} integer from \emph {value}
to \emph {address} in a non-cached fashion.

\begin{procedure}{CPU\_WRITE\_VECTOR}{3}
	\pin{mode}{VectorMode}{Integer 8, 16, 32, 64; Float; Double}\\
	\pin{to}{access <VectorMode>}{}\\
	\pout{from}{access <VectorMode>}{}\\
	\pin{size}{Integer32}{}
\end{procedure}

This procedure writes a vector of \emph{size}, \emph{to} and reads
it \emph{from}. This operation is executed in a specific \emph{mode}.

\skipoff
\section {Interrupts management}
This section presents the elements that allow to deal with interrupts.
\skipon

\type{interrupt\_id\_t}

This type represents an interrupt line of a processor or of a processor's
subsystem.

\type{interrupt\_status\_t}

This type represents the interrupt status of a processor or of a processor's
subsystem.

\type{interrupt\_handler\_t}

This type represents the handler of a processor interrupt.

\begin{procedure}{CPU\_IT\_ENABLE}{1}
	\pin{vector}{interrupt\_id\_t}{}
\end{procedure}

This procedure masks the interruption indexed by \emph{vector}.

\begin{procedure}{CPU\_IT\_DISABLE}{1}
	\pin{vector}{interrupt\_id\_t}{}
\end{procedure}

This procedure unmasks the interruption indexed by \emph{vector}.

\begin{procedure}{CPU\_IT\_ATTACH\_ISR}{2}
	\pin{vector}{interrupt\_id\_t}{}\\
	\pin{handler}{interrupt\_handler\_t}{}
\end{procedure}

This procedure attaches an interrupt \emph{handler} to a specific
interrupt \emph{vector}.

\skipoff
\section {Exception management}
This section presents the elements that allow to deal with processor exceptions.
\skipon

\type{exception\_id\_t}

This type represents an exception of a processor.

\type{exception\_handler\_t}

This type represents the handler of a processor exception.

\begin{procedure}{CPU\_IT\_ATTACH\_ESR}{2}
	\pin{vector}{exception\_id\_t}{}\\
	\pin{handler}{exception\_handler\_t}{}
\end{procedure}

This procedure attaches an exception \emph{handler} to a specific
exception \emph{vector}.

\skipoff
\section {Multiprocessor management}
This section presents the elements that allow to deal multiprocessor
configurations.
\skipon

\begin{function}{CPU\_MP\_COUNT}{Integer32}{0}
\end{function}

This function returns the number of CPUs identical to the one calling
the function.

\begin{function}{CPU\_MP\_ID}{Integer32}{0}
\end{function}

This function returns the unique processor's identifier.

\begin{procedure}{CPU\_MP\_WAIT}{1}
	\pin{sync}{access Integer32}{}
\end{procedure}

This procedure spins until the value of the variable \emph{sync} turns 0.
This variable must exist and must be initialized to 1 beforehand.

\begin{procedure}{CPU\_MP\_PROCEED}{1}
	\pin{sync}{access Integer32}{}
\end{procedure}

This procedure sets the variable \emph{sync} to 0.

\skipoff
\section {Synchronization management}
This section presents the elements that allow to deal with synchronization.
\skipon

\begin{function}{CPU\_TEST\_AND\_SET}{Integer32}{1}
	\pin{lock}{access Integer32}{}
\end{function}

This procedure performs a \emph{test-and-set} operation on the given
\emph{lock}.

\begin{function}{CPU\_COMPARE\_AND\_SWAP}{Integer32}{1}
	\pinout{lock}{Integer32}{}\\
	\pin{old}{Integer32}{}\\
	\pin{new}{Integer32}{}
\end{function}

This procedure performs a compare-and-swap on the specified \emph{lock}, using
the two \emph{old} and \emph{new} values as reference.
